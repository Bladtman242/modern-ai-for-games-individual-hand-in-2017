\section{Agent: FSM and Evolutionary Computation}
\label{sec:GA}

The first agent (henceforth GA) was submitted to the first part of the competition, and consists of three parts:

\begin{enumerate}
\item Three pre-implemented strategies for the agent.
\item A state machine that transitions between pre-implemented strategies based
	on game state.
\item A genetic algorithm that determines threshold values for the transitions
	in the state machine.
\end{enumerate}

The GA agent hence has two "phases". First; a learning phase where the genetic
algorithm finds good parameters for the state transitions. This happens before
the game is played, and is not part of the agents actual behaviour.

The second phase is the actual play, where a strategy is determined by the state
machine -- based on the parameters found by the genetic algorithm -- and played out.
\subsection{The pre-implemented strategies}
The three pre-implemented strategies are as follows:

 \texttt{FIND\_PILL} strategy, in which the agent determines the location of the
nearest pill in the maze, and follows the shortest path to it.

A \texttt{FLEE} strategy, in which the shortest-path lengths between the agent
and all potentially threatening ghosts\footnotemark\ are determined.
Potentially threatening ghosts are considered \textit{actual} threats if they
are less than a certain distance from the agent. This distance threshold is
determined by the genetic algorithm. The agent makes a move away from all
threatening ghosts if possible, otherwise the agent makes a random valid move.

A \texttt{HUNT\_GHOST} strategy, in which the agent determines the location of
the closest edible ghost (again, by shortest-path length), and moves towards
that position.

\textbf{TODO:} describe genetic algorithm
\textbf{TODO:} remaining sections for GA

\footnotetext{A ghost is considered a potential threat if it is not
blue, and is not inside the ghost lair.}

\subsection{Description of Approach (e.g. input, output, action, state space,
representation)}

\subsection{Algorithm Parameters:}

\subsection{Which parameters needed to be adjusted (e.g. ANN learning rate),
which were picked by default (e.g. connection weight interval) and why?}

\subsection{What was the experiment followed to adjust those parameters of the
algorithm.}

\subsection{Performance measure of the algorithm}

\subsection{Experiments (My player against the provided fixed-players)}
Conclusions (Why did it work? Why it didn't work? If it didn't work as expected what could you suggest that it has to be changed for the algorithm to work and why didn't you try it?)
