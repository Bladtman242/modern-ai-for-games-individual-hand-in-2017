\section{Abstract}
In this report three different artificial intelligences for Ms. Pac-Man are
presented and evaluated. Ms. Pac-Man is a real-time arcade game with no clear
terminal state. Unlike other games, such as chess, there is no state in which
the player has won or lost the game. Rather, each game state can be described by certain metrics such as score, level reached, or time survived. Optimizing any
combination of these and other metrics one might be interested in, can be
chosen as the goal of game. The player competes with four opponent ghosts. In
general, the ghosts' strategies can be of arbitrary complexity, ranging from
random to collaborative play.

These properties of Ms. Pac-Man are prohibitive. The combination of decisions
being real-time, the game having a total of five agents (four ghosts and
Pac-Man), indeterministic ghost play, and the state-space being practically
infinite and lacking a clear goal state, makes the design and implementation of
a Pac-Man agent non-trivial.

\section{Introduction}
This report describes the implementation of three Ms. Pac-Man agents:
\begin{enumerate}
	\item A finite state automata whose transitions are determined by a genetic
		algorithm (section \ref{sec:GA}).
	\item An agent trained with temporal difference learning (section
		\ref{sec:qlearning}).
	\item An agent controlled by monte carlo tree search (section \ref{sec:mcts})
\end{enumerate}
